%%
%    * ----------------------------------------------------------------
%    * "THE BEER-WARE LICENSE" (Revision 42/023):
%    * Ronny Bergmann <mail@rbergmann.info> wrote this file. As long as
%    * you retain this notice you can do whatever you want with this
%    * stuff. If we meet some day and you think this stuff is worth it,
%    * you can buy me a beer or a coffee in return.
%    * ----------------------------------------------------------------
%
%
% A german example using the Kartei.cls - including print and toc as
% options, hence all pages are Din A4.
%
% Last Change: Kartei 1.9, 2012/01/04
%
\documentclass[a6paper,10pt,grid=front%
,toc
%,print
]{kartei}
\usepackage[utf8]{inputenc} %UTF8
\usepackage{hyperref}
\begin{document}
  \setcardpagelayout

  \section*{ANS08}

  \begin{karte}{Definition Data Warehouse}    
    Ein Data Warehouse dient dazu, Daten aus unterschiedlichen internen und externen Quellen zusammenzuführen und zu speichern, um anschließend mithilfe unterschiedlicher Abfrage-, Analyse- und Auswertungsprogrammen neue Informationen zu gewinnen.
  \end{karte}

  \begin{karte}{Worin besteht der Unterschied zwischen operativen \& analytischen Daten?}
    \includegraphics[width=0.9\textwidth]{img/diff_operativ_analytische_daten}
  \end{karte}

  \begin{karte}{Ziele eines Data Warehouse?}
    \begin{itemize}
      \item Informationen für das Management
      \item Unterstützung von Entscheidungen
      \item zusammenführen unterschiedlicher Daten aus operativen Anwendungssystemen
      \item es werden (un-)strukturierte Daten übernommen
      \item Veränderung, Aggregation der Daten
    \end{itemize}
  \end{karte}

  \begin{karte}{Aufbau analytischer Informationssysteme}
    \begin{itemize}
      \item \textbf{Zentrales DWH} enthält eine von den operativen Systemen isolierte Datenbank
      \item \textbf{Data Mart} ist ein subjektspezifisches oder abteilungsspezifisches DWH; entweder Datenbestände gleichzeitig an mehreren Orten schneller bereitzustellen oder einzelne Fachabteilungen spezifische Daten zu liefern
    \end{itemize}
    \includegraphics[height=0.5\paperheight]{img/aufbau_analytischer_is}    
  \end{karte}

  \begin{karte}{Unterschied Data Warehouse \& Data Mart}
    \includegraphics[width=0.9\paperheight]{img/diff_dwh_dm}
  \end{karte}

  \begin{karte}{Ablauf ETL?}
    \begin{itemize}
      \item Analyse und Dokumentation operativer und externer Datenquellen
      \item Extrahieren der ausgewählten Daten
      \item Transformation operativer Daten
      \item Bereinigung transformierter Daten
      \item periodisches Laden der Daten ins DWH
    \end{itemize}
  \end{karte}

  \begin{karte}{Extraktion}
    Unter Extraktion versteht man die Selektion der Daten aus den (zumeist) operativen Datenquellen und ihre anschließende Speicherung in einen Arbeitsbereich des DWH (Staging Area). Hier werden die Daten zwischengespeichert und transformiert bzw. bereinigt und im Anschluss in das DWH übertragen.
  \end{karte}

  \begin{karte}{Wann wird die Extraktion durchgeführt?}
      \begin{itemize} 
      \item Periodisch
      \item Anfrage
      \item Ereignisgesteuert (wenn z.B. Werte unterschritten werden)
      \item Sofort (DWH hat die gleiche Aktualität wie die operativen Systeme)
    \end{itemize}
  \end{karte}

  \begin{karte}{Transformation}
      Transformation findet in der s.g. Staging Area statt und bereinigt bzw. transformiert die Quelldaten in das gewünschte Zielformat.
  \end{karte}

  \begin{karte}{Qualitätsmängel der Quelldaten}
    \begin{itemize} 
      \item inkorrekte Daten (Eingabe-/Verarbeitungsfehler)
      \item logisch widersprüchliche Daten
      \item unvollständige, ungenaue, zu grobe Daten
      \item redundante Daten
      \item uneinheitliche Daten
      \item veraltete Daten
      \item irrelevante Daten
      \item unverständliche Daten (wegen qualitativ mangelhafter Metadaten)
    \end{itemize}

    \textbf{Verfahren:}
    \begin{itemize}
      \item Bereinigung
      \item Harmonisierung (betriebswirtschaftlich: Codierung, Schlüssel, Attribute)
      \item Verdichtung (für Analysezwecke aggregiert werden $\rightarrow$ Regionalzahlen usw.)
      \item Anreicherung (Ergänzung um errechnete Kennzahlen)
    \end{itemize}
  \end{karte}

  \begin{karte}{Bereinigung - Was ist zu beachten?}
    \begin{itemize} 
      \item Muss-Feld?
      \item Plausibilitätsprüfung bei der Eingabe?
      \item Wir das Feld gemäß der ursprünglichen Bestimmung genutzt?
      \item Wurde das Datenfeld nachträglich aufgenommen? (fehlt bei älteren Daten dann)
      \item Existieren konkrete Änderungspläne für die operativen Daten?
    \end{itemize}    
  \end{karte}

  \begin{karte}{Daten-Mängel}
    Es werden \textbf{syntaktische} und \textbf{semantische} Mängel unterschieden.

    \includegraphics[width=0.9\textwidth]{img/maengel}
  \end{karte}

  \begin{karte}{Harmonisierung - Was wird getan?}
    \begin{itemize}
      \item Vereinheitlichung unterschiedlicher Codierungen (z.B. männlich, m, 1, weiblich, w, 0)
      \item Synonyme und Homonymen (unterschiedliche Attributnamen mit gleicher Bedeutung z.B. vorname, vname, firstname)
      \item Harmonisierung von Schlüsseln und Kennzahlen      
    \end{itemize}
  \end{karte}

  \begin{karte}{Verdichtung}
    Es werden Daten im DWH (Staging Area) auf verschiedenen Stufen aufsummiert.
    \includegraphics[width=0.9\textwidth]{img/verdichtung}
  \end{karte}

  \begin{karte}{Anreicherung}
    Es werden Berechnungen durchgeführt, die zusammen mit den übrigen analytischen Daten gespeichert werden, d.h. es werden konkrete Kennzahlen ermittelt basierend auf einem gegeben Kennzahlensystem (z.B. DuPont-Schema $\rightarrow$ ROI)

    Vorteile der Anreicherung sind:
    \begin{itemize}
      \item kürzere Antwortzeiten bei späteren Anfragen da es sich um vorberechnete Werte handelt
      \item hohe Datenkonsistenz, da sie nach einem einheitlichen Algorithmus berechnet werden
    \end{itemize}
  \end{karte}  

  \begin{karte}{Laden}
    Es wird unterschieden zwischen:
    \begin{itemize}  
      \item Initialem Füllen aus den operativen Datenbanken
      \item Zyklischer Aktualisierung, neue Werte werden ergänzt, alte archiviert
    \end{itemize}

    Wenn die Daten zyklisch übernommen werden kann dies als:

    \begin{itemize}
      \item Kompletter Abzug (einfach aber zeitaufwendig)
      \item jeweilige Änderungen (geringe Datenmenge, aufwendig das Delta zu ermitteln, nur der letzte Stand wird ermittelt)
      \item Auswahl protokollierter Datenbanktransaktionen (auch Änderungen innerhalb des Deltas erfasst werden)
    \end{itemize}

    geschehen.
  \end{karte}

  \begin{karte}{Metadaten}
    Metadaten sind Daten über Daten und enthalten Hintergrundinformationen über die im DWH gespeicherten Werte. Sie geben Aufschluss über:
    \begin{itemize}
      \item Umfang der verfügbaren Daten
      \item Datenstruktur und Beziehungen (Relationen)
      \item Herkunft der operativen Daten
      \item Speicherort im DWH
      \item Formate
      \item Zugriffsberechtigungen      
    \end{itemize}

    In der Metadatenbank wird festgehalten:

    \begin{itemize}
      \item Welche Daten woher kommen
      \item Wie sie aufbereitet und verdichtet werden
      \item Wo sie gespeichert werden
      \item Welcher Anwender auf welche Daten Zugriff erhält
    \end{itemize}
  \end{karte}

  \begin{karte}{Archivierung}  
    Es wird zwischen der:

    \begin{itemize}
      \item Datenarchivierung (auslagern auf Offlinedatenträger nach Zeit)
      \item Datensicherung (Dienen zur Wiederherstellung des DWH)
    \end{itemize}

    unterschieden.
  \end{karte}

  \begin{karte}{OLAP}  
    Der Begriff steht für Online Analytical Processing und umfasst alle Formen der \textit{mehrdimensionalen Datenanalyse}. Im Focus stehen betriebswirtschaftliche Kennzahlen. Die Mehrdimensionalität wir durch s.g. \textit{Datenwürfel} veranschaulicht.

    \includegraphics[height=0.65\paperheight]{img/wuerfel}
  \end{karte}

  \begin{karte}{Unterschied OLAP/TLTP}  
    \includegraphics[width=0.9\textwidth]{img/diff_olap_oltp}
  \end{karte}

  \begin{karte}{Anforderungen an OLAP}  
        
    \begin{itemize}
      \item Mehrdimensionale konzeptionelle Sicht auf die Daten (Zeit, Produktgruppe, Region, Person, usw.)
      \item Transparenz (Anwender müssen keine technischen Details kennen)
      \item Zugriffsmöglichkeiten (auf möglichst viele heterogene und interne/externe Datenquellen)
      \item Stabile Antwortzeiten (möglichst schnell und vor allem gleichbleibend)
      \item Client-/Server-Architektur
      \item Gleichrangige Dimensionen
      \item Dynamische Handhabung \"dünn besetzter\" Matrizen (effiziente Speicherung trotz Lücken)
      \item Mehrbenutzerfähigkeit
      \item Unbeschränkt dimensionsübergreifende Operationen
      \item Intuitive Datenanalyse
      \item Flexibles Berichtswesen (Dokumentation in FOrm von Berichten und Grafiken)
      \item Unbegrenzte Anzahl von Dimensionen/Aggregationsstufen
    \end{itemize}

    \textbf{Kritik:} Die unscharfe Trennung zwischen fachlich-konzeptionellen Anforderungen und technischer Realisierung.

    Alternativ \textbf{FASMI}:

    \begin{itemize}
      \item Fast (Antwortzeit max. 20s)
      \item Analysis (Anwender ohne technisches Wissen müssen auswerten können)
      \item Shared (Mehrbenutzer)
      \item Multidimensional
      \item Information (sämtliche benötigten Informationen können geliefert werden)      
    \end{itemize}
  \end{karte}

  \begin{karte}{Mehrdimensionalität}  
    Die Anzahl der Dimensionen lässt sich mit der Fakultät der Dimensionen berechnen:
    2 Dim = 1 x 2 = 2 Sichten\\
    3 Dim = 1 x 2 x 3 = 6 Sichten\\
    4 Dim = 1 x 2 x 3 x 4 = 24 Sichten\\

    Die verschiedenen Betrachtungsmöglichkeiten werden auch als \textbf{Slice and Dice} bezeichnet. \textit{Slice} bedeutet das Herausschneiden von Scheiben aus dem Würfel. \textit{Dice} bedeutet die Bildung von kleinen Würfeln aus dem Gesamtwürfel zur Einschränkung auf einen Wert bzw. Wertebereich.

    Mittels \textbf{Drill down} ist es möglich von einer bestehenden Verdichtungsebene auf eine detaillierte Ebene zu wechseln. \textbf{Drill up} wechselt von einer Ebene auf eine verdichtertere Ebene. \textbf{Drill across} ermöglicht zu einem anderen Wert auf der selben Ebene zu wechseln.
  \end{karte}

  \begin{karte}{Wie können Daten verdichtet werden?}  
    
    Wie die Daten verdichtet werden können hängt unmittelbar vom Typ ab.

    \begin{itemize}
      \item \textbf{Additive Daten} lassen sich beliebig aufsummieren (Umsatz in Kombination mit Produkten und Regionen)
      \item \textbf{Semiadditive Daten} lassen sich nicht über alle Dimensionen aufaddieren (z.B. bei Zeiträumen und Lagerbeständen)
      \item \textbf{Nichtadditive Daten} lassten keine sinnvolle Aufsummierung zu (Anteilswerte)
    \end{itemize}
  \end{karte}

  \begin{karte}{Modellierung}    
    Die \textbf{multidimensionale Modellierung} ist in der Lage die konzeptionellen Datenstrukturen in physische Datenbank-/Speicherstrukturen umzusetzen. Im Falle eine \textit{dünn besetzen Matrix} wird Speicherplatz vergeudet, da für nicht existente Werte gleich viel Speicher verbraucht wird, als wenn der Wert existiert.

    Die \textbf{relationale Modellierung} setzt die mehrdimensionale Datenstruktur in einer relationalen Datenbankstruktur ab. Es wird dabei zwischen:

    \begin{itemize}
      \item Fakten (quantitative Kerndaten)
      \item Dimension (beschreibende Daten, kann beliebig viele Ausprägungen haben)
    \end{itemize}

    unterschieden.
  \end{karte}

  \begin{karte}{Star-Schema}  
    Für jede Dimension wird eine Tabelle eingerichtet. Die Dimensionstabellen sind nicht miteinander verknüpft, sondern stehen nur über die Faktentabelle miteinander in Beziehung. Der Primärschlüssel in der Faktentabelle setzt sich zusammen aus den Primärschlüsseln aller Dimensionstabellen.

    Vorteil:
    \begin{itemize}
      \item einfaches, intuitive Datenmodell
      \item benötigt nur wenige Join-Operationen
      \item wenige Tabellen benötigt
    \end{itemize}

    Nachteil:
    \begin{itemize}
      \item durch die große Anzahl der Verknüpfungsmöglichkeiten
kann es zu Performanceproblemen kommen
    \end{itemize}
    \includegraphics[height=0.45\paperheight]{img/star_schema}
  \end{karte}

  \begin{karte}{Snowflake-Schema}  
    Dieses Schema ist einer Erweiterung des Star-Schemas und wird durch die Normalisierung entwickelt.

    Vorteile:
    \begin{itemize}
      \item erleichtert die Aggregation
      \item keine Redundanzen      
    \end{itemize}

    Nachteile

    \begin{itemize}
      \item höhere Anzahl JOINS
      \item höhere Anzahl Tabellen
      \item komplexere SQL-Abfragen
    \end{itemize}

    \includegraphics[height=.45\paperheight]{img/snowflake}
  \end{karte}

  \begin{karte}{Was ist Data-Mining?}
    \textit{Data-Mining} bedeutet die Erforschung und Analyse großer Datenbestände hinsichtlich sinnvoller Muster, Zusammenhänge oder Regeln, mit dem Ziel sinnvolle Informationen heraus zu filtern. Die Besonderheit ist, das verborgene und bisher unbekannte Sachverhalte ermittelt werden sollen.

    Eine alternative Vorgehensweise ist das \textit{Hypothesis Testing} bei dem Annahmen bestätigt oder widerlegt werden. Im Idealfall generiert das Data-Mining-System die Hypothesen selbst und überprüft diese indem die Datenbank nach Mustern und Zusammenhängen durchsucht werden.
  \end{karte}

  \begin{karte}{Merkmale/Hauptanwendungsgebiete von Data-Mining}          
    Merkmale:
    \begin{itemize}
      \item Selbstständigkeit (soll autonom arbeiten)
      \item Mustergenerierung
      \item Interessantheit (Ergebnisse müssen von Bedeutung sein)
    \end{itemize}

    Anwendungsgebiete:
    \begin{itemize}
      \item Vorhersagen unbekannter oder zukünftiger Werte
      \item Entdeckung bislang unbekannter Zusammenhänge
      \item automatisierte Analyse und Klassifizierung
    \end{itemize}
  \end{karte}

  \begin{karte}{Alternative Verfahren zur Datenmusterkennung}
    Beim \textit{Text-Mining} wird in unstrukturierten Datenbeständen gesucht. Problematisch ist die geringe innere Struktur der durchsuchten Dokumente.
    
    Beim \textit{Web-Mining} werden Daten durchsucht die durch die Nutzung des WWW entstanden sind (z.B. Logdateien). Problematisch ist das viele Zugriff über Proxies erfolgen, was eine Anonymisierung zur Folge hat.    
  \end{karte}

  \begin{karte}{Methoden des Data-Mining}  
    Die \textit{Klassifikation} ordnet einzelne Objekte (Datensätze) vorgegeben Klassen zu. Diese Zuordnung erfolgt anhand bestimmter und bekannten Eigenschaften (Attributwerten).

    Die \textit{Clusterbildung} soll ähnliche Objekte zu Gruppen (Segmente) zusammenfassen. Die Differenz zur Klassifikation besteht darin, das die verwendeten Merkmale und die Bereiche ihrer Ausprägung nicht im Voraus festgelegt werden.

    Die \textit{Assoziation} sucht nach Verwandtschaft und Korrelation innerhalb des Datenbestands. Das Verfahren ermittelt Regeln über Zusammenhänge zwischen einzelnen Merkmalen und bewertet die gefundenen Muster mit Wahrscheinlichkeiten. Es entstehen s.g. Wenn-Dann-Regeln.
  \end{karte}

  \begin{karte}{Welche Faktoren beeinflussen die Übersichtlichkeit von Informationen?}  

    \begin{itemize}
      \item Format (Hoch-/Querformat $\rightarrow$ Präsentation/Fließtext)
      \item Gliederung (Inhaltsverzeichnisse, Absätze, Überschriften)
      \item Darstellungsformen (Schaubilder)
    \end{itemize}

    Zur Visualisierung werden häufig auch s.g. Metaphern verwendet.

    \begin{itemize}
      \item Landkarten-Metapher (Typisch im Marketing und Vertrieb)
      \item Wetterkarten-Metapher (Regionale Trends und Entwicklungen)
      \item Organigramm-Metapher (Kennzahlensysteme)
      \item Zeitungs-Metapher (Stark strukturierte Informationen)
      \item Unternehmensleitstand-Metapher (aktuelle Stati, Dashboard)
    \end{itemize}
    
  \end{karte}

  \begin{karte}{Welche Funktionen haben Kennzahlen?}  
        
    \begin{itemize}
      \item Vorgabefunktion als Zielgröße für Teilbereiche
      \item Anregungsfunktion zur Erkennung von Auffälligkeiten und Veränderungen
      \item Steuerungsfunktion zur Vereinfachung von Steuerungsprozessen
      \item Kontrollfunktionen zur laufenden Erfassung von Soll/Ist-Abweichungen
      \item Vergleichsfunktion (Benchmarking)
      \item Informationsfunktion in Frühwarnsystemen oder Berichtserstattung
    \end{itemize}
  \end{karte} 

  \begin{karte}{Wie können Kennzahlen unterschieden werden?}  
    Nach der \textit{Reichweite}: unternehmensspezifisch, branchenbezogen, allgemeingültig

    Nach der \textit{zeitlichen} Gültigkeit: vergangenheits-, gegenwarts-, zukunftsbezogen

    Nach der \textit{Art}: statisch (Summen, Differenzen, Mittelwerte), relativ (Gliederungszahlen, Beziehungszahlen, Indexzahlenx)
  \end{karte}

  \begin{karte}{Welche Berichte können unterschieden werden und was ist essentiell für diese?}  
    Es wird zwischen \textit{außerbetriebliche} (Geschäftsbereicht) und \textit{innerbetrieblichen} Berichten unterschieden.

    Für beide ist es entscheidend, dass der Bericht:

    \begin{itemize}
      \item zum richtigen Zeitpunkt
      \item in der richtigen Detaillierung
      \item in der richtigen Form
      \item für die richtige Adressatengruppe
    \end{itemize}

    zur Verfügung gestellt wird.
  \end{karte}

  \begin{karte}{Welche Berichtsarten können unterschieden werden?}  
        
    \begin{itemize}
      \item Standardbericht: periodisch, gleicher Aufbau
      \item Bedarfsbericht: Inhalt, Zeiträume und Formate sind Parameterisierbar und er wird fallweise angefordert
      \item Ausnahmeberichte: zeigen Abweichungen vom Normalzustand
    \end{itemize}
  \end{karte}

  \begin{karte}{Was ist ein Führungsinformationssystem (FIS)?}  
    FIS werden auch als Executive Information Systems (EIS) bezeichnet. Für die Zielgruppe ist eine intuitive Bedienung eine managmentgerechte Informationspräsentation notwendig. FIS werden unternehmensspezifisch aufgebaut. Die Anwender sollten selbstständig in der Lage sein das System zu bedienen.

    \textit{Exception Reporting} soll frühzeitig auf Abweichungen vom Sollzustand aufmerksam machen.
  \end{karte}

  \begin{karte}{Historie Computerarbeit}  
    Seit der Einführung der Computer in den Unternehmen hat eine Dezentralisierung stattgefunden. Zuerst musste der Computer-Nutzer nur einfache Masken bedienen, mit der Einführung des PCs in den Büros jedoch musste er zunehmend selbstverantwortlich mit dem Computer arbeiten. Speziell seit der Einführung von Tabellenkalkulationsprogrammen stehen ihm zum Teil eher Werkzeuge zum Selbstlösen von Aufgaben zur Verfügung als fertige Lösungen. Mit der Vernetzung der PCs und der wachsenden Bedeutung des Internets sowie der daraus resultierenden Gefahren stieg zudem die Verantwortung, da Fehlbedienungen an einem Rechner auch Auswirkungen auf andere Arbeitsplätze haben können.    
  \end{karte}

  \begin{karte}{Organisation von IT-Abteilungen}  
    Innerhalb der Organisation eines Unternehmens unterscheidet man die Fachabteilungen sowie die IT-Abteilung(en). Für die organisatorische Eingliederung der IT-Abteilungen gibt es sehr unterschiedliche Varianten, die unter anderem von der Größe und technischen Ausrichtung des Unternehmens abhängen. Bei kleineren, eher nichttechnischen Unternehmen wird entweder ein vorhandener Mitarbeiter teilweise mit solchen Aufgaben betraut oder auch ein kleiner IT-Bereich geschaffen. Größere und grundlegende IT-Aufgaben werden in der Regel nach außen vergeben. Größere Unternehmen verfügen fast immer über eine eigene, zum Teil intern weiter strukturierte IT-Abteilung. Das schließt jedoch nicht aus, dass auch hier einzelne Tätigkeitsbereiche extern vergeben werden (Outsourcing). Bei sehr großen oder sehr technikorientierten Unternehmen werden oft zusätzlich bereichsspezifische IT-Abteilungen installiert.
  \end{karte}

  \begin{karte}{Welcher rechtliche Rahmen gilt?}  
    Unternehmen müssen bei der Nutzung von Computern zahlreiche rechtliche Vorschriften beachten. Die wichtigsten sind:
        
    \begin{itemize}
      \item Vermeidung von illegalen Software-Kopien
      \item Einbeziehung des Betriebsrates beim Einsatz von Computern
      \item Einhaltung der Vorschriften des Datenschutzes
      \item Langfristige Aufbewahrung und das Zugänglichmachen steuerlich relevanter gespeicherter Daten
      \item Vorbeugende Vermeidung rechtswidriger Handlungen der Mitarbeiter sowie Schädigung Dritter
      \item Regelmäßige Datensicherungen nach dem Stand der Technik
    \end{itemize}
  \end{karte}

  \begin{karte}{Was ist bei der Auswahl von Software zu beachten?}  
    Bei der Auswahl von Office-Systemen sind folgende Komponenten zu berücksichtigen, die sich gegenseitig beeinflussen bzw. bedingen:

    \begin{itemize}
      \item Hardware (vor allem PC-Systeme und Mac)
      \item Betriebssystem (vor allem Windows, Linux und Mac OS)
      \item Anwendungssoftware
    \end{itemize}

    Bei der Entscheidung für ein System sind folgende Kriterien anzulegen: Sofern vorhandene Hard- und Software oder auch Schnittstellen zu anderen Stellen berücksichtigt werden müssen, ist unbedingt auf entsprechende \textbf{Kompatibilität} zu achten.

    Beim \textbf{Leistungsumfang} sind zunächst die Systeme auszuschließen, die die K.-o.-Kriterien nicht erfüllen. Bei den verbleibenden Produkten empfiehlt sich eine Nutzwertanalyse, bei der die Leistungen in einzelnen Bereichen mit ihrer Wichtigkeit für die geplante Anwendung gewichtet werden. Neben den Eigenschaften der Produkte selbst ist auch deren Marktposition und die Unterstützung durch Dritte zu beachten.

    Bei den Kosten ist nicht der Anschaffungspreis entscheidend. Vielmehr müssen die \textbf{Total Cost of Ownership} berücksichtigt werden, die unter anderem auch Arbeitskosten durch Installation, Schulung und Umstellung vorhandener Dokumente o.Ä. beinhalten.
  \end{karte}

  \begin{karte}{Wie können Office-Anwendungen integriert werden?}  
    Die Integration von Officeanwendungen bezieht sich auf zwei getrennte Aspekte: Zum einen geht es um die \textbf{Integration der Benutzeroberfläche}. Die einzelnen Programme besitzen dabei nicht nur eine einheitliche Gestaltung und Menüstruktur, sondern bilden im Idealfall eine einheitliche Applikation, in der die Grenzen zwischen den einzelnen Anwendungen für den Benutzer weitgehend aufgehoben sind.

    Zum anderen geht es um die \textbf{Integration der Dokumente} unterschiedlicher Applikationen auf Daten- bzw. Dateiebene. In Windows wird dazu OLE verwendet, das zwei Integrationstechniken umfasst:

    Beim \textbf{Verknüpfen} (Linking) bleiben die Dokumente in getrennten Dateien gespeichert und es wird im Zieldokument nur eine Referenz (Link) auf die Datei des Quelldokuments gespeichert. Vorteile sind dabei der geringe Speicherplatz und die ständige Aktualität. Nachteilig sind unbeabsichtigte Auswirkungen, wenn die Dokumente unabhängig voneinander inhaltlich oder bezüglich ihres Speicherplatzes verändert werden.

    Beim \textbf{Einbetten} (Embedding) wird eine Kopie des Quelldokuments in die Datei des Zieldokuments eingefügt. Der Vorteil besteht darin, dass auf diese Weise Änderungen mit unbeabsichtigten Auswirkungen auf das Zieldokument ausgeschlossen sind und bei Dateioperationen die Einheit der Dokumente immer erhalten bleibt. Als Nachteil sind der (zum Teil zusätzliche) Speicherplatz innerhalb des Zieldokuments sowie die fehlende automatische Anpassung bei Aktualisierungen des originalen Quelldokuments zu nennen.
  \end{karte}

  \begin{karte}{Wie können Office-Anwendungen konfiguriert werden?}  
    Heutige Officeprogramme erlauben dem Nutzer vielfältige Konfigurationsmöglichkeiten, mit denen er das Programm seinen individuellen Bedürfnissen anpassen kann. Dabei unterscheidet man zwei Gruppen von Einstellungen:

    \textbf{Programmeinstellungen} beziehen sich auf das Aussehen bzw. Verhalten des Programms selbst. Beispiele sind das Aussehen von Symbolleisten bzw. Menübändern sowie der Speicherort von Dateien.

    In \textbf{Dokumentvorlagen} werden Einstellungen gespeichert, die sich auf neu erstellte Dokumente auswirken. Typische Vorgaben sind hier die Schriftart und Einstellungen für Ränder. Neben einer Standardvorlage, die automatisch für neu erstellte Dokumente verwendet wird, kann der Benutzer eine beliebige Anzahl weiterer Dokumentvorlagen für verschiedene Zwecke erstellen und beim Erzeugen eines neuen Dokuments auswählen.
  \end{karte}

  \begin{karte}{Wie läuft das mit der Wartung?}  
    Viele Programme müssen regelmäßig oder aufgrund bestimmter äußerer Ereignisse aktualisiert werden, um noch ausreichend nutzbar zu bleiben. Solche Aktualisierungen müssen vom Hersteller bezogen werden, der diese auf unterschiedliche Art zur Verfügung stellen kann:


    \begin{itemize}
      \item Fehlerbeseitigungen und das Schließen von Sicherheitslücken sollten grundsätzlich kostenlos erfolgen (in der Regel per Download aus dem Internet).
      \item Regelmäßige Aktualisierungen und Funktionserweiterungen werden – insbesondere im professionellen Bereich – über Wartungsverträge angeboten. Das gilt bei Unternehmen inzwischen auch für Office-Produkte.
      \item Mit dem Erscheinen einer neuen Version eines Programms stehen zum Teil neue Funktionen zur Verfügung, die einen Umstieg sinnvoll erscheinen lassen. Hier ist – sofern nicht über einen entsprechenden Wartungsvertrag abgedeckt – der Neukauf dieser Version notwendig, wobei es üblicherweise einen Rabatt für Besitzer der Vorversion(en) gibt.
    \end{itemize}
  \end{karte}

  \begin{karte}{Was ist individuelle Informationsverarbeitung?}  
    Unter dem Begriff der Individuellen Informationsverarbeitung (IIV) versteht man, dass der Benutzer an seinem lokalen Arbeitsplatz mit dort verfügbaren Werkzeugen selbstständig Endbenutzeraufgaben mit arbeitsplatzbezogenen Daten wahrnimmt. Trotz der Bezeichnung \"individuell\" ist dies im betrieblichen Bereich in die Informations-Infrastruktur des Unternehmens eingebunden und damit stark standardisiert. Etwas einschränkender ist der Begriff der Officeanwendungen, der sich auf die Anwendungen im Büro beschränkt, wie sie über alle Branchen und Funktionen hinweg am Computer anfallen. Typische Programme, die auch den Kern von Officepaketen bilden, sind Textverarbeitung, Tabellenkalkulation und Präsentationsgrafik.
  \end{karte}

  \begin{karte}{Was wird unter kooperativer Informationsverarbeitung verstanden?}  
    Die computerunterstützte Gruppenarbeit wird allgemein mit dem Begriff Computer Supported Cooperative Work (CSCW) bezeichnet. Unter Groupware versteht man die Systeme, mit denen dies unterstützt wird. Es werden meist zwei Schwerpunkte unterschieden:
      
    \textbf{Workflow Computing} betrachtet arbeitsteilige Prozesse mit einer Abfolge einzelner Tätigkeiten, die meist von verschiedenen Personen in einer definierten Reihenfolge abgearbeitet werden.
    \textbf{Workgroup Computing} ist weniger strukturiert und stellt die Gruppe und ihre Kooperationsbeziehungen in den Mittelpunkt.

    Eine weitere Unterscheidung bezieht sich einerseits auf den \textbf{Ort}, an dem sich die Teilnehmer aufhalten (gleicher Ort oder verschiedene Orte), andererseits auf den \textbf{Zeitbezug} der gemeinsamen Arbeit (synchron oder asynchron). Für den Austausch von Informationen existiert innerhalb eines Betriebes heute fast immer ein lokales Netzwerk (LAN), das die PCs miteinander mit hoher Geschwindigkeit verbindet.
  \end{karte}

  \begin{karte}{Was ist ein Dokumentenmanagementsystem?}  
    Dokumentenmanagementsysteme (DMS) dienen dazu, Dokumente in elektronischer Form zu erfassen, bereitzustellen, weiterzuleiten und dauerhaft zu archivieren.

    Beim \textbf{Erfassen} der Dokumente werden diese in ein geeignetes elektronisches Format überführt, wobei man kodierte und nicht kodierte Informationen unterscheidet.
    
    Bei der \textbf{Indizierung} werden den Dokumenten für die Speicherung Metainformationen hinzugefügt.

    Bei der \textbf{Speicherung} der Daten ist ein schneller Zugriff ohne menschliches Zutun sicherzustellen. Wichtig ist außerdem die Vergabe angemessener Zugriffsrechte sowie die Absicherung gegen Datenverlust.

    Mit der \textbf{Recherche} werden Dokumente nach bestimmten Kriterien gesucht. Dies geschieht über die bei der Indizierung vergebenen Metainformationen oder eine Volltextsuche.
    
    Bei der \textbf{Verwendung} der gefundenen Dokumente kann man die Verarbeitung mit dem Originalprogramm des jeweiligen Formats oder einem speziellen Viewer unterscheiden.
  \end{karte}

  \begin{karte}{Welche Arten von Datensicherung gibt es?}  
    Unter Datensicherung (Backup) versteht man das Speichern von Daten, die noch aktuell auf dem Arbeitssystem gehalten werden, auf einem Sicherungsmedium. Datensicherungen müssen regelmäßig durchgeführt und die Sicherungsmedien besonders geschützt verwahrt werden.
    
    Bei der \textbf{Vollsicherung} werden sämtliche relevanten Daten zu einem Zeitpunkt gesichert.
  
    Bei der \textbf{inkrementellen Sicherung} werden nur die Daten gesichert, die sich seit der letzten Sicherung geändert haben.
    
    Bei der \textbf{differenziellen Sicherung} werden nur die Daten gesichert, die sich seit der letzten Vollsicherung geändert haben.
  \end{karte}
\end{document}