\documentclass[a6paper,11pt,print,grid=front]{kartei}

\usepackage[ngerman]{babel}
\usepackage[utf8]{inputenc} %UTF8

\begin{document}

\begin{karte}[Lebensphilosophie]{Wie lautet die Antwort auf die Frage nach dem Leben dem Universum und dem Ganzen Rest ?}[prüfungsrelevant]
42
\end{karte}

\begin{karte}[Zahlenkunde]{Was ist der Unterschied in der Verwendung von Drölf und $n$ bei Ihnen?}
$n$ wird verwendet für Zahlen bis hin zu verdammt groß, Drölf nur bis hin zu verdammt.
\end{karte}
	
\section*{Informatik}
\subsection*{Spaß mit Verweisen}

%Den Kommentar im Stil ändern
\renewcommand{\kommentarstil}{\textsc}

\begin{karte}{Was ist verschränkte Rekursion ?}[ein Beispiel für Label]
\label{karte:antwort} Siehe Karte \ref{karte:frage}
\end{karte}

\begin{karte}{Was ist die Antwort auf Karte \ref{karte:antwort} ?}
\label{karte:frage}		Hier kommt man eigentlich gar nicht hin. Hier gibt es also nichts zu sehen, bitte blättern sie unauffällig weiter.
\end{karte}

\end{document}